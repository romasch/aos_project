\documentclass[a4paper,10pt]{article}
\usepackage[utf8]{inputenc}


% For the \todo{} command.
\usepackage{todo}

% Nice fonts
\usepackage{palatino}
% For doubled lines in tables.
% Useful for separation of table header from table body.
\usepackage{hhline}
\usepackage[english]{babel}
\usepackage{tabularx}
% Needed for Listings package with Eiffel.
% \usepackage{xcolor}
% Source code listings.
\usepackage{listings}
% Appendix with extra title.
% \usepackage [page] {appendix}
% To include PNG files.
% \usepackage{graphicx}
% Nice looking captions.
\usepackage[font={footnotesize,sl}, labelfont=bf] {caption}
% Include PDF pages.
% \usepackage{pdfpages}

% Clickable links. Has to be the last package:
\usepackage [hidelinks] {hyperref}


\lstset{language=C,basicstyle=\ttfamily\small}


\newcommand{\todoref}{\todo{ref}}
\newcommand{\filepath}[1]{\emph{ #1}}

% Title Page
\title{Advanced Operating Systems \\ Project Report}
\author{Roman Schmocker \\ Yauhen Klimiankou}


\begin{document}

\maketitle


\section{Introduction}

This report provides documentation for the operating system which we built as part of the project in the Advanced Operating Systems course.
The code is based on a stripped-down version of Barrelfish \cite {web:barrelfish}.

\todo{ more introduction}

\section{Getting started}

\begin{lstlisting}

void test_listings_package (void);

void test_listings_package (void)
{
  uint32_t an_int;
  char* a_string = "asdf";
  for (int i=0; i<42; i++) {
    printf (a_string);
  }
}
\end{lstlisting}


\section {Modules}

\subsection{Paging}

% 
% - Intent: quick lookup of page table, ``unmap'' requests from memory server
% 
% - Data structues
% -- Simple for virtual memory
% -- 2-level array for page tables (mirroring actual page tables)
% -- linked list for frames
% 
% - Problem: No unmap from domain itself
% -- esp. no recycling of virtual addresses
% -- we thought that malloc would never return memory anyway
% -- didn't think about domain spawning / bulk transfer with explicit map/unmap operations

The paging code is mostly contained in \filepath{lib/barrelfish/paging.c} and the corresponding header file.
The central element is the \lstinline!paging_state! struct.
When desgining the data structure for the paging code we wanted to support the following functions in constant time:

\begin{itemize}
 \item Check if an address is valid.
 \item Lookup of the page tables associated to the virtual address.
 \item Allocation of a physical frame for a page.
 \item Deallocation of a frame when the memory server asks for it.
\end{itemize}

To support the first use case we just keep track of the allocated range of virtual addresses using two integer variables.
That way determining if a page fault is ``valid'' is very simple: 
If it's in the range, everything is ok and we can map the page to a frame, and if not we abort due to an invalid address.

The data structure of choice to represent the page table structure is a two-level array that mirrors the actual page tables in memory.
Currently we only store the capability to the page tables and whether a certain page has been allocated at some point.
We don't yet store any flags, pointers to a swap file or other metadata, but it would be easy to add this information.

Frame management is a bit more complex because we always allocate frames of 1 MiB.
That way we can avoid an IPC call for every single page fault, but we also have to keep track of free and allocated space within a frame.
The \lstinline!struct frame_list! manages this information for a single frame.
To support the last use case it also keeps track of the actual pages stored on this frame.
The \lstinline!paging_state! struct then maintains a double ended queue of \lstinline!frame_list! nodes, where only the frame at the head of the list may contain some free space.

As already mentioned, an important use case we had in mind were requests to return a frame from the memory server.
We never implemented that functionality, but the data structure is designed to support it.
An implementation would have to do the following:
\begin{itemize}
 \item Remove the last \lstinline!frame_list! element.
 \item Write all pages backed by this frame to disk.
 \item (Optionally) update some flags and metadata in the page table array.
 \item Return the frame to the memory server.
\end{itemize}

The data structures used to track frames and page tables need some form of dynamic memory management.
Unfortunately however we can't use malloc/free in our paging implementation, because this memory is not initially mapped.
Therefore we used slab allocators (from \filepath{include/barrelfish/slab.h}) to manage the memory needed by these data structures.
The feature \lstinline!memory_refill! in our paging code is used to enlarge the pool of usable memory if the slab allocators run out of space.

\subsubsection{Assessment}

Our initial design of the paging module turned out to be rather flexible.
Over the course of the project, we only had to change small things, like adding another slab allocator for exception stacks of user-level threads 
or implementing \lstinline!paging_map_fixed_attr!, where we could adapt an existing function that did almost the same job.

The only thing that proved to be insufficient was the management of virtual address space.
The initial assumption that only malloc/free can issue unmap operations (and never does as it's not implemented) was wrong.
There are some parts in the code that directly interacts with the paging library and needs to ``free'' a range of virtual memory sometimes, among them the spawndomain library or the bulk transfer mechanism.

Supporting this would have required us to rewrite a lot of existing code, and that's why we didn't do it in the end.
This in turn means that our current solution leaks virtual addresses, but we tried to reduce this as much as possible by reusing the ranges of virtual memory.

\subsection{Local Message Passing}

The local message passing (LMP) system is mostly implemented in \filepath{lib/barrelfish/aos\_rpc.c} for the client side.
Some generic server-side features are implemented in \filepath{lib/aos\_support/server.c}.

The LMP system has some important characteristics:
\begin{itemize}
 \item All messages are synchronous, i.e. receive is blocking.
 \item The channels are always one-to-one connections.
 \item The communication is client-server style.
 \item We support direct connections to any server (i.e. no indirection through init).
 \item Every ``send'' message has an enum constant as its first argument that denotes the message type.
 \item Every ``reply'' message has an error code as its first argument.
 \item All other arguments are determined by the actual message type.
\end{itemize}

The individual message types and their arguments are described in the header file \filepath{include/barrelfish/aos\_rpc.h}.
To support writing code on the client side we implemented the feature \lstinline!aos_send_receive! which takes a struct containint all arguments, 
sends a request to the server, waits for a reply, and copies the receive arguments back into the struct.
That way we could avoid having to write the same functionality for every message type.

Some servers provide core system services. Init for exampleprovides RAM, and the serial\_driver provides I/O.
We therefore provide some predefined channels to these services which are initialized in \filepath{lib/barrelfish/init.c}.
The channels can be retrieved with the functions \lstinline!aos_rpc_get_init_channel! and \lstinline!aos_rpc_get_serial_driver_channel!, respectively.

\subsubsection{Connection setup and name service}

In our LMP implementation we wanted to support direct channels between two domains.
To do that we had to implement some additional functionality, in particular a simple name service and a mechanism to set up a new connection.

The design of the name service is very simple.
In \filepath{aos\_rpc.h} we added an enum type \lstinline!aos_service! which is used for some of the core system services, such as the serial driver or the FAT file system.
Each domain that acts as a server for such a core system service then has to register itself via the \lstinline!AOS_RPC_REGISTER_SERVICE! LMP message.
Init itself keeps track of all registered servers and therefore acts as a name service.

When a client wants to find a particular service, it sends a request to init, and init replies with a new endpoint capability of the specified domain.
This was a bit tricky to implement, because we only allow one-to-one connections, so init first has to request a new endpoint from the (registered) server.
Init shouldn't mix the reply however with some other request from another domain, therefore we had to break our usual send-reply protocol and plan this interaction carefully.

In chronological order, this is what happens when a domain wants to find, for example, the serial driver:

\begin{itemize}
 \item The client sends an \lstinline!AOS_RPC_FIND_SERVICE! with the enum constant of the serial driver to init.
 \item Init uses a new request ID (a ``cookie'') and remembers the channel that sent the request.
 \item Init checks if the serial driver is registered and sends an \lstinline!AOS_ROUTE_REQUEST_EP!.
 This call includes the previously generated request ID.
 \item The serial driver handles the enpoint request call by creating a new channel.
 It replies to init with an \lstinline!AOS_ROUTE_DELIVER_EP! which contains the new endpoint and the unchanged request ID as arguments.
 Note that this RPC call doesn't follow the usual send-reply style message exchange, because the reply contains a message type argument.
 \item Thanks to the special reply format init can handle the request like any other LMP message.
 When receiving the \lstinline!AOS_ROUTE_DELIVER_EP!, init retrieves the channel that initially sent the \lstinline!AOS_RPC_FIND_SERVICE! request using the request ID and forwards the endpoint.
 \item The client domain wakes up again receives the new endpoint, which it can then use to establish a direct connection to the serial driver.
\end{itemize}


% - Synchronous message passing, one-to-one
% - channels between any client/server possible (not just init)
% - general format: service identifier, args -> reply: error code, args
% - aos\_rpc.c: client API
% -- support feature \lstinline!aos_send_receive!
% -- only initialize arguments and receive result.
% - aos\_support/server.c: generic server support routines
% - routing: in init - service registration, find requests
% - bulk transfer: shared buffer, bound to a channel
% - predefined channels (provided by libbarrelfish): init, serial\_driver


\subsubsection{Bulk transfer}

For a long time throughout the project we only had the basic message-passing communication model.
Although this worked well, we always had to split an IPC call into several messages if the data to be transmitted exceeded the capacity of a single message.
This was a tedious and error prone process, and due to this we decided to add a mechanism for bulk transfer.

The basic idea is to set up a buffer which is shared between client and server.
This setup is initiated by the client, and only when it is actually needed.

We added a new message type which allowed a client to share a frame with a buffer and get back a \lstinline!memory_descriptor!, which it can use afterwards for other IPC calls.
The server-side part is implemented in \filepath{lib/aos\_support/}, which had the nice side effect that every server gained the buffer sharing mechanism for free without any further changes.
The client-side part is implemented in \filepath{lib/barrelfish/aos\_rpc.c}, where it is now possible to ``attach'' a shared buffer to any \lstinline!struct aos_rpc! channel.

The bulk transfer mechanism was a pretty late addition to the code base, therefore not all RPC calls have been changed to make use of it yet.


\subsection{Shell}
	Shell is a standard mean of communication and interaction between user and developed operating system.
	In the same time it is the only user interface currently existing in OS.
	 
	Historically shell was grew up in the context of memeater module and in some moment almost completely preempt its original functionality. The source code can be find at path \filepath{usr/memeater/memeater.c}.
	
	The main difference of the our shell from the conventional shells is that in fact it is remote shell which uses the host system as an source of input and target for output streams. 
	Communication with the host computer system goes through serial port interface (also knows as RS-232) for which incoming stream of characters is treated as a standard input stream and outgoing set of characters is treated as a standard output stream in terms of C language. 
	Host computer system is such design plays o role of conventional terminal by connecting the input stream stream of characters from the serial port with display and pushing the characters inputted from keyboard to the same serial port.
	Picocom utility is in use for maintaining of this bridge on the host Linux-based host.
	
	Communication via serial port is implemented in separate server called serial driver., located at path \filepath{usr/serial\_driver/}.
	Shell interacts with this driver using functions from the set of standard IO library of C and AOS RPC calls \lstinline!aos_rpc_serial_getchar! and \lstinline!aos_rpc_serial_putchar!.
	
	Like most of the servers execution of the server consist of three phases:
	\begin{enumerate}
		\item Initialization.
		\item Service loop.
		\item Deinitialization.
	\end{enumerate}
	During the initialization phase shell sets up its global state and establishes connections with other system components which will be used in the service loop.
	The set of system components used by shell includes:
	\begin{enumerate}
		\item Filesystem driver.
		\item Led driver.
		\item Process manager.
		\item Serial diver.
	\end{enumerate}
	Global state of the shell consist of:
	\begin{enumerate}
		\item Sign of termination.
		\item String describing the current directory.
	\end{enumerate}
	
	Service loop of the shell is conditionally endless loop which is run until the sign of termination will not be set. 
	Each iteration of this loop handles one input request an d consist of several steps:
	\begin{enumerate}
		\item Assembling of the request string.
		\item Splitting of the assembled request into two parts: command and arguments.
		\item Look up of the executor.
		\item Execution of the request.
	\end{enumerate} 
	Purpose of the first step is to split endless input stream coming from the serial port via serial driver into separate strings each of which represent single request (implemented in function \lstinline!receive_request!).
	Carriage return character of ASCII encoding table is used as a separator between sequential requests.
	Function \lstinline!receive_request! also reflects each incoming character back to the serial port.
	This reflection allows user on the host see what he types on keyboard in real time and by this control correctness of input and reply for the inputted request from the shell running on Pandaboard.
	Unfortunately, Picocom doesn't support the special handling of the control characters and by default displays all characters disregard to its nature.
	As a result typo in the inputting command can be fixed only by flushing the current wrong request and retyping it again.
	Once the complete request is assembled it is passed to the second step of service loop iteration.
	  
	To facilitate request processing it splits into two parts -- command and arguments.
	Command is considered as a first word of request where arguments is a rest of the request.
	Space characters between command and arguments are omitted.
	At the current state shell doesn't make splitting of the arguments string into separate argument words.
	Instead, it is responsibility of each command executor to split and process arguments in appropriate way. 
	
	  Shell support two types of commands:
	  \begin{enumerate}
	  	\item Internal
	  	\item External
	  \end{enumerate}  
	  Internal commands implementations are embedded into the shell itself. List of internal commands can be found in table \ref{tbl:icl}.
	  External commands are expected to be implemented as a separate processes and shell expect that appropriate binaries are part of the Barrelfish boot image.
	  Internal commands have a priority over external commands.
	  As a result, there is no way to execute external command with the same name as one of the internal commands have.
	  Internal commands look up implemented as a simple loop through the internal command table until entry with the same name won't be found. 
	  
	\begin{table}[h]
       		\centering
        	\begin{tabular}{| m{2.0cm} | m{1.5cm} | m{7.0cm} |}
			\hhline{===}
            		\textbf{Command name} 	& \textbf{Args} 	& \textbf{Description}	\\
			\hhline{===}
            		cat 				& file path 							& Display the content of a file specified as an argument. 																\\ \hline
            		cd 				& --- \newline file path \newline \newline	& Sets the current directory to root \newline	Set the current directory in accordance with absolute or relative path specified in the argument	.	\\ \hline
            		echo 				& string 							& Display a string specified as an argument																			\\ \hline
            		exit 				& --- 							& Force termination of shell																						\\ \hline
            		kill 				& integer 							& Force termination of the process which PID specified as an argument. Init process (PID 0) can't be terminated.						\\ \hline
            		ledoff			& ---								& Turn off the led on the Pandaboard.																				\\ \hline
            		ledon				& --- 							& Turn on the led on the Pandaboard.																				\\ \hline
            		ls 				& --- 							& Display list of files and directories from the current directory.															\\ \hline
            		oncore			& integer + string 					& Execute request specified as a string on kernel which id specified as an integer. Can be used to execute programs on slave kernel.			\\ \hline
            		ping 				& --- 							& Send series of `Ping' messages to the `test\_domain' process. Expects that last one in the running state.							\\ \hline
            		ps 				& --- 							& Display the currently-running processes.																			\\ \hline
            		run\_memtest		& integer 							& Run memory allocation test, which allocates and fills memory chunk of specified size. 											\\ \hline
            		test\_string		& --- 							& Send very long predefined string to the serial driver and by this display it.													\\ \hline            		
        	\end{tabular}
        	\caption{List of shell internal commands}
        	\label{tbl:icl}
	\end{table}
	
	External commands can be executed in two modes: background and foreground.
	By default shell execute commands on the foreground mode, but execution of the command in the background mode can be forced by putting the `\&' character at the end of request.
	Command running in the foreground mode suspends the shell until command executor process will be terminated and intercepts control over serial port channel.
	Command running in background mode runs concurrently with the shell and without access to the serial port channel.
    	    	
	 \section{Serial I/O}
	 	The serial driver is a mediator between serial port connection and the rest of the rest of operating system. It is located in \filepath{usr/serial\_driver/}.
	 	It have straightforward implementation due to the simplicity of the serial port management at least on the level of use which we have.
	 	Driver didn't support buffering, interrupts, baud rate control etc. and relies to the preliminary port serial port initialization made by kernel during boot up.
	 	
	 	Handling of the serial port is is based on polling.
		
		Driver provides the next set of services:
		\begin{enumerate}
			\item Put char. Output character into serial port.
			\item Get char. Input character from the serial port.
			\item Put string. Output a string of characters to the serial port. This makes use of the bulk transfer mechanism to avoid splitting and transmitting the string in small chunks.
			\item Set foreground. Set the process id of the process which will be the only process eligible to receive input from the serial port.
		\end{enumerate}
		
		In addition for convenience the libc was customized to use serial port driver as back end for the prinf and scanf families of functions.
		As a result any application is able to use traditional C-like IO.

\subsection{Process management}
\label{sec:process-management}
- currently done in init
- server manages process table and info about loaded modules
-- can be printed with ps command
- shell attempts to start program with specific name if command not recognized
- mechanism to register for end notification (foreground tasks)
- kill command to revoke dispatcher capability (background tasks)
- no further cleanup for zombie domains as for now

\subsection{Mupltiple Cores}

\subsubsection{Preparing the kernel image}

The preparation of the kernel image is mostly done in \filepath{usr/init/cross\_core\_setup.c}.
The implementation is heavily inspired\footnote{copy-pasted} from the upstream Barrelfish code base.
Compared to the upstream mechanism we implemented some simplifications however, since we don't have to support more than two cores.
We also adapted the code a bit to our own data structures.

\subsubsection{Low-level mechanisms}

To start the second core one has to call the \lstinline!sys_boot_core! system call, which eventually executes \lstinline!start_aps_arm_start! in the first kernel.
Within this function there's no magic going on - it just writes the start address for the second core to a regsiter and issues an interrupt.

The real magic happens in \lstinline!arch_init! in \filepath{kernel/arch/omap44xx/init.c} and  \lstinline{arm_kernel_startup} in \filepath{kernel/arch/omap44xx/startup\_arch.c}.
In \lstinline!arch_init! we set up the \lstinline!global! struct which contains data to be shared between both kernels.
Among the data shared between the two kernels is the multiboot image.
We also reserve a region of physical memory as a buffer for future cross core communication and store the physical address and size of it in the \lstinline!global! struct.

In \lstinline!arm_kernel_startup! we do a static split of the available physical memory.
Each init thus only gets a capability to half of the memory, which prevents them from messing around with each others address space.
Finally, we also create the capability to the shared frame in this functions and store it in the \lstinline!TASKCN_SLOT_MON_URPC! slot.

\subsubsection{Cross core communication}



\todo {}

- Spawning cores on low level.
- Setup of ELF image
- Communication between cores

\subsection{FAT file system}

The FAT filesystem is implemented in \filepath{aos\_support/fat32.c}, whereas the server and disk driver is located in \filepath{usr/mmchs/}.
We only support read operations on a FAT filesystem, and we didn't try to improve the block driver as well.

The main data structure for the FAT filesystem is the config struct:

\begin{lstlisting}
struct fat32_config {
    sector_read_function_t read_function;

    uint32_t volume_id_sector;
    uint32_t sectors_per_cluster;
    uint32_t fat_sector_begin;
    uint32_t cluster_sector_begin;
    uint32_t root_directory_cluster;
};
\end{lstlisting}

Within the struct we store some of the most important data, such as the start address of the File Allocation Table or the cluster region.
We also store a pointer to a function needed to read a block from disk.
That way we can easily reuse the FAT driver module for block devices other than the SD card of the Pandaboard.

A common operation in the FAT filesystem is to follow a chain of clusters and read consecutive sectors.
We added a small support structure \lstinline!struct sector_stream! to hide this complexity.
A sector stream is an iterator over a file or directory and can be advanced with \lstinline!stream_next!.

\subsubsection{File management}

Open files are identified by a file descriptor.
When opening a file we scan the directories in the path until we find a file with a matching name, and then we store the start cluster, file size, and the \lstinline!fat32_config! structure.






- filesystem code in aos\_support/fat32.c
- handler in mmchs/filesystem\_server.c
- Currently read-only
- some support for partition tables
- Performance improvements: sequential read? cached FAT? bulk transfer?

\subsubsection{ ELF loading}

\todo {Implement ELF loading}

\section{Conclusion}

\begin{flushleft}
{{{
\bibliographystyle {plain}
\bibliography {./references}
}}}
\end{flushleft}


\todos

\end{document}          
 
