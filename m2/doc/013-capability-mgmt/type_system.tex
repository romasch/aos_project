%%%%%%%%%%%%%%%%%%%%%%%%%%%%%%%%%%%%%%%%%%%%%%%%%%%%%%%%%%%%%%%%%%%%%%%%%%%%%%%%
\chapter{Type system}\label{chap:type_system}

In this chapter, we cover the type model of capabilities and the
supported types in Barrelfish.

%%%%%%%%%%%%%%%%%%%%%%%%%%%%%%%%%%%%%%%%%%%%%%%%%%%%%%%%%%%%%%%%%%%%%%%%%%%%%%%%
\section{Type Model}
\note{We do not implement capability rights yet.}
    
\begin{description}
\item[Name] Each type has a unique name.

\item[Origin] A capability type is either \emph{primitive}, which
  means that capabilities of the type may be created only through a
  special process (eg. at boot time), or \emph{derived}, which means
  that capabilities of the type may be created by retyping an existing
  capability. For primitive types, we specify how the capabilities of
  that type are created; for derived types, we specify which
  capability types may be retyped to yield a capability of the given
  type.

\item[Retypability] Some types of capability may be \emph{retyped} to
  create one or more new capabilities of the same or different
  type. If this is the case, we specify for each type from what other types of
  capability it may be retyped.

\item[Mint parameters] It is possible to specify type-specific
  parameters when \emph{minting} capabilities. We specify for each
  type the interpretation of the type-specific parameters. When not
  specified, they are ignored.

\item[Interpretation of rights] The interpretation of the primitive
  capability rights is type-specific. A capability type defines the
  interpretation of these rights, usually in the specification of the
  legal invocations.

\item[Transferability to another core] Depending on its type, it may
  or may not be possible to transfer a capability to another core.

\item[Last copy deleted] The type specific operations to perform when
  the last copy of a capability is deleted. For capability types
    that refer to actual memory, if the last reference to a piece of
    memory is deleted, then the memory must be garbage collected.

\item[Concrete representations] Each capability type has one or more
  representations: in the memory of each core on which it may appear,
  and in a canonical serialised representation used for transferring
  it in a message between cores. These are specified as
  Hamlet\cite{hamlet} data types.

\item[Invocations] Most capability types support one or more
  type-specific operations through the use of the invoke system call.
  (documented in \ref{sec:sys_invoke}).
\end{description}

%%%%%%%%%%%%%%%%%%%%%%%%%%%%%%%%%%%%%%%%%%%%%%%%%%%%%%%%%%%%%%%%%%%%%%%%%%%%%%%%
\section{Types}
  
\subsection{CNode}\label{sec:cnode}

A CNode refers to an array of capabilities of some power-of-two size.
CNodes are used to hierarchically construct the CSpace of a domain, as
described in \ref{sec:cspace}.  All capability management is
performed through invocations on CNodes.

A CNode capability stores a \emph{guard} and \emph{guard size}, which
is expressed as a number of bits. As in guarded page
tables\cite{Liedtke_GPT}, the guard allows the depth of the CSpace
tree to be reduced by skipping levels that would only contain a single
mapping. When resolving a CSpace address that has led to a CNode
capability, the guard is compared with the corresponding number of
bits from the remaining part of the address, and if it does not match,
the lookup fails.

Many CNode invocations require that the caller provide both a CSpace
address and the number of \emph{valid bits} in the address. This
allows the invocations to refer to another CNode capability that is
located at an intermediate level in the tree, and thus would usually
be recursed through by the address resolution algorithm. If the number
of valid bits associated with a CSpace is less than the size of a full
CSpace address, only the least significant bits that are valid, but
starting with the most significant bit thereof, are used, and the
address lookup terminates early.

\begin{description}
\item[Origin] Retyping from RAM type capabilities

\item[Retypability] No

\item[Mint parameters] The mint parameters can be used to set a guard
  on a CNode.
  \begin{itemize}
  \item Parameter 1: The guard to set.
  \item Parameter 2: The size of the guard in bits.
  \end{itemize}

\item[Interpretation of rights] \note{Explain rights and rights mask.
  Capability rights and rights masks are currently not implemented.
  This means that every user domain holding a capability has full
  rights to it.}

\item[Transferability to another core] Yes.  When transfered to
  another core, capability is implicitly retyped to a Foreign CNode
  type. \note{We do not allow CNode type caps to be transferred yet.}

\item[Last copy deleted] When the last copy is deleted, all
  capabilities stored within it are also deleted.

\item[Concrete representations] The in-memory representation of on
  x86-64 is as follows:
    
  \begin{lstlisting}[language=Mackerel]
    datatype cnode_cap "CNode capability" {
      cnode 64 "Physical base address of CNode";
      bits 8 "Number of bits this CNode resolves";
      rightsmask 8 "Capability rights mask";
      guard_size 8 "Number of bits in guard word";
      guard 32 "Bitmask already resolved when reaching this CNode";
    };
    \end{lstlisting}
\end{description}

\begin{invocation}{Mint}\label{sec:mint}
  \arg CSpace address of destination CNode
  \arg Slot number in destination CNode
  \arg CSpace address of source capability
  \arg Number of valid bits in destination CNode address
  \arg Number of valid bits in source capability address
  \arg Type-specific parameter 1
  \arg Type-specific parameter 2
\end{invocation}
The Mint invocation creates a new capability in an existing CNode
slot, given an existing capability.  The new capability will be a copy
of the existing capability, except for changes to the
\emph{type-specific parameters}.

The use of the two type-specific parameters is described along with
the description of the relevant type.

If the destination capability is of type VNode, then instead of a
mint, a page table entry is made into the page table pointed to by the
VNode.  In this case, the source capabilities must be of type VNode,
Frame, or Device Frame.  The first parameter specifies architecture
specific page (table) attributes and the second parameter specifies an
offset from the source frame/device frame to map.
  
\begin{invocation}{Copy}
  \arg CSpace address of destination CNode
  \arg Slot number in destination CNode
  \arg CSpace address of source capability
  \arg Number of valid bits in destination CNode address
  \arg Number of valid bits in source capability address
\end{invocation}
This invocation is similar to Mint, but does not change any
type-specific data.

If the destination capability is of type VNode, then instead of a
copy, a page table entry is made into the page table pointed to by the
VNode.  In this case, the source capabilities must be of type VNode,
Frame, or Device Frame.

\begin{invocation}{Retype}
  \arg CSpace address of source capability to retype
  \arg Type of new objects to create
  \arg Size in bits of each object, for variable-sized objects
  \arg CSpace address of destination CNode
  \arg Slot number in desination CNode of the first created capability
  \arg Number of valid bits in destination CNode address
\end{invocation}

This invocation creates one or more new descendant capabilities of the
specified type in the specified slots, given a source capability and a
destination type.  It will fail if the source or destination are
invalid, or if the capability already has descendants. (some
capability types, currently only the dispatcher type can be retyped
even if it already has descendants).  The destination slots must all
occupy the same CNode.  The permissible source/destination pairs are
shown in \ref{fig:cap_types} and \ref{tab:retype_types}.  The number
of new capabilities created is determined by the size of the source
capability divided by the size of the newly-created objects.

\ctable[
  caption=Permissible types for the Retype invocation,
  label=tab:retype_types,
]{lll}{}{
  \FL Source & Destination & Variable size?\ML
  Physical address range & Physical address range & Yes\NN
  Physical address range & RAM & Yes\NN
  Physical address range & Device frame & Yes\NN
  RAM & RAM & Yes\NN
  RAM & CNode & Yes\NN
  RAM & VNode & No\NN
  RAM & Dispatcher & No\NN
  RAM & Frame & Yes\NN
  Dispatcher & IDC endpoint & No\NN
}

\begin{figure}
  \centering
  \includegraphics[width=.7\columnwidth]{cap_types}
  \caption{Valid capability retyping paths}\label{fig:cap_types}
\end{figure}
  
\begin{invocation}{Delete}
  \arg CSpace address of capability to delete
  \arg Number of valid bits in capability address
\end{invocation}
This invocation deletes the capability at the given address, freeing
the associated CNode slot.

\begin{invocation}{Revoke}
  \arg CSpace address of capability to Revoke
  \arg Number of valid bits in capability address
\end{invocation}
This invocation revokes the capability at the given address.

The capability itself is left untouched while all its descendants and
copies are deleted.

\subsection{Foreign CNode}
\note{This has not been implemented yet}

The foreign CNode capability gives a domain on a core the ability to
specify a capability that actually resides on another core.  This
capability allows for the holder to create local copies of the
capabilities stored in the actual CNode modulo rights as can be
implemented.  The capability tracks on which core the actual CNode
resides.  \note{Full implementation and discussion pending}

\begin{description}
\item[Origin] When a CNode capability are copied to another core.

\item[Retyability] No

\item[Mint parameters] None
  
\item[Interpretation of rights] \note{Explain rights}
  
\item[Transferability to another core] Yes

\item[Last copy deleted] \note{NYI}
  
\item[Concrete representations] The in-memory representation on x86-64 is as follows:
  
  \begin{lstlisting}[language=Mackerel]
    datatype fcnode_cap "Foreign CNode capability" {
      cnode      64  "Physical base address of CNode";
      bits        8  "Number of bits this CNode resolves";
      rightsmask  8  "Capability rights mask";
      core_id     8  "Core id of the core the actual CNode capability
                      resides in";
      guard_size  8  "Number of bits in guard word";
      guard      32  "Bitmask already resolved when reaching this CNode";
    };
  \end{lstlisting}
\end{description}

\note{Discussion pending on invocations}

\subsection{Physical address range}

Most domains will generally not handle capabilities of this type.
They are introduced because the kernel relies on the user-space
software to decide the location of RAM.

By retyping physical address range capabilities to RAM, the caller
guarantees that the underlying region does contain RAM that can safely
be used for storage of kernel objects.  Any domain with access to
physical address range capabilities is therefore a critical part of
the trusted computing base.

\begin{description}
\item[Origin] Created at boot time in the bsp core based on the
  multiboot info.

\item[Mint parameters] None
  
\item[Retyability] To Physical address range, RAM or DevFrame type.
  
\item[Interpretation of rights] \note{Explain rights}
  
\item[Transferability to another core] Yes

\item[Last copy deleted] \note{NYI, maybe inform some special
  dispatcher like memory server}
  
\item[Concrete representations] The in-memory representation on x86-64 is as follows:
  
  \begin{lstlisting}[language=Mackerel]
    datatype physaddr_cap "Physical address range capability" {
      base       64  "Physical base address of region";
      bits        8  "Size of region";
    };
  \end{lstlisting}
\end{description}

\subsection{RAM}

A RAM capability refers to a naturally-aligned power-of-two-sized
region of kernel-accessible memory.

\begin{description}
\item[Origin] Retyping from physical address range capabilities.
  
\item[Retyability] To RAM, Frame, CNode, VNode, or Dispatcher types.
  
\item[Mint parameters] None
  
\item[Interpretation of rights] \note{Explain rights}
  
\item[Transferability to another core] Yes

\item[Last copy deleted] \note{NYI, maybe inform some special
  dispatcher like memory server}
  
\item[Concrete representations] The in-memory representation on x86-64 is as follows:
  
  \begin{lstlisting}[language=Mackerel]
    datatype ram_cap "RAM capability" {
      base       64  "Physical base address of region";
      bits        8  "Size fo region";
    };
  \end{lstlisting}
\end{description}

\subsection{Dispatcher}
This capability type refers to the kernel object associated with a
user-level dispatcher (see \ref{sec:dispatch}).

\begin{description}
\item[Origin] Retyping from RAM capabilities.
  
\item[Retyability] To IDC Endpoint type
  
\item[Mint parameters] None
  
\item[Interpretation of rights] \note{Explain rights}
  
\item[Transferability to another core] No

\item[Last copy deleted] \note{NYI, maybe inform some special
  dispatcher like spawn daemon}
  
\item[Concrete representations] The in-memory representation on x86-64
  is as follows:
  
  \begin{lstlisting}[language=Mackerel]
    datatype dcb_cap "Dispatcher capability" {
      dcb        64  "Pointer to the in kernel representation of
                      the dispatcher control block";
    };
  \end{lstlisting}
\end{description}

\begin{invocation}{Setup}
  \arg CSpace address of domain's root CNode (root of CSpace)
  \arg Number of valid bits in root CNode address
  \arg CSpace address of domain's root VNode (root page table)
  \arg Number of valid bits in root VNode address
  \arg CSpace address of dispatcher frame (user-level dispatcher
  data) capability
  \arg Whether to make dispatcher runnable
\end{invocation}
This invocation sets any of the above parameters on a dispatcher
object.  If any of the CSpace addresses are null, they are ignored.
Additionally, once all of the parameters are set (either in a single
invocation, or after multiple invocations), and if the runnable flag
is set, the dispatcher is made runnable.  \note{There are additional
  invocations in the code that we have not discussed yet.}

\subsection{IDC Endpoint}
    
Every IDC endpoint refers both to a dispatcher and an \emph{endpoint
  buffer} within that dispatcher. The endpoint buffer is specified as
an offset from the start of the dispatcher frame, and is the location
where the kernel delivers IDC messages. It is also delivered to the
user with an LRPC message.  The initial endpoint offset of an IDC
endpoint capability when it is retyped from a dispatcher capability is
zero; the capability cannot be used to send IDC until the the offset
is specified changed by minting an endpoint to another endpoint.

\begin{description}
\item[Origin] Retyping Dispatcher type capabilities.

\item[Mint parameters] The mint parameters can be used to change the
  badge on the capability
  \begin{itemize}
  \item Parameter 1: The endpoint offset to set on the capability.
  \end{itemize}
  
\item[Retyability] No
  
\item[Interpretation of rights] \note{Explain rights}
  
\item[Transferability to another core] No

\item[Last copy deleted] \note{NYI, inform some entity to initiate
  connection teardown}
  
\item[Concrete representations] The in-memory representation on x86-64
  is as follows:
  
  \begin{lstlisting}[language=Mackerel]
    datatype idc_cap "IDC endpoint capability" {
      listener     64  "Pointer to the in kernel representation of the
                        receiver's dispatcher control block";
      epoffset     64  "Offset of endpoint buffer within dispatcher
                        structure";
    };
  \end{lstlisting}
  
\item[Invocation] Any invocation of an endpoint capability causes the
  entire message to be delivered to the dispatcher to which the
  endpoint refers (see \ref{sec:IDC}).
  \end{description}

\subsection{VNode}
A VNode capability refers to a hardware page table and is used to
manage a domain's virtual address space.  Frame and device frame
capabilities can be copied or minted into them or deleted from them by
invoking a CNode.  The architecture may impose limitations on the
capabilities that may be copied into a VNode, or may allow extra
attributes to be set when minting; see \ref{apx:arch_specific}.

\begin{description}
\item[Origin] Retyping from RAM type capabilities.

\item[Retyability] No

\item[Mint parameters] None
  
\item[Interpretation of rights] \note{Explain rights}
  
\item[Transferability to another core] \note{Discussion pending}

\item[Last copy deleted] \note{NYI, initiate mechanisms to unmap from
  associated page tables and remove mapped in page tables and frames}
  
\item[Concrete representations] The in-memory representation on x86-64
  is as follows:
  
  \begin{lstlisting}[language=Mackerel]
    datatype vnode_cap "VNode capability" {
      base     64  "Base address of the page table";
    };
  \end{lstlisting}
\end{description}  

\subsection{Frame}
A frame capability refers to a naturally-aligned power-of-two-sized
region of physical memory that may be mapped into a domain's virtual
address space (by copying it to a VNode).  When a frame capability is
created (ie.~retyped from RAM), the kernel zero-fills the frame.
\note{Is this a good idea? Shouldn't we be able to pre-zero frames?
  -AB}

\begin{description}
\item[Origin] Retyping from RAM type capabilities.
  
\item[Retyability] To Frame type
  
\item[Mint parameters] None
  
\item[Interpretation of rights] \note{Explain rights}
  
\item[Transferability to another core] Yes

\item[Last copy deleted] \note{NYI, initiate unmapping from page
  tables.  We may choose for this to happen when the last copy of a
  frame within a dispatcher is deleted rather than the last copy in
  the entire system.}
  
\item[Concrete representations] The in-memory representation on x86-64
  is as follows:
  
  \begin{lstlisting}[language=Mackerel]
    datatype frame_cap "Frame capability" {
      base       64  "Physical base address of untyped region";
      bits        8  "Size of the region";
    };
  \end{lstlisting}
\end{description}  

\noarginvocation{Identify}
This invocation returns the physical address and size (in bits) of the frame.

\subsection{Device frame}
A device frame capability refers to a naturally-aligned
power-of-two-sized region of physical address space that may be mapped
into a domain's virtual address space (by copying it to a VNode).
Unlike frame capabilties, the kernel does not zero-fill device frame
capabilities upon mapping.  As the name implies, device frames are
typically used for access to memory-mapped devices.

\begin{description}
\item[Origin] Retyping Physical address range type capabilities.
  
\item[Retyability] To Device frame type
  
\item[Mint parameters] None
  
\item[Interpretation of rights] \note{Explain rights}
  
\item[Transferability to another core] Yes

\item[Last copy deleted] \note{NYI, initiate unmapping from page
  tables.  We may choose for this to happen when the last copy of a
  frame within a dispatcher is deleted rather than the last copy in
  the entire system.}
  
\item[Concrete representations] The in-memory representation on x86-64 is as follows:
  
  \begin{lstlisting}[language=Mackerel]
    datatype device_cap "Device Frame capability" {
      base       64  "Physical base address of region";
      bits        8  "Size of the region";
    };
  \end{lstlisting}
\end{description}  

\noarginvocation{Identify} This invocation returns the physical
address and size (in bits) of the frame.

\subsection{IO}
IO capability gives the holder the ability to read and write to IO
ports.

\begin{description}
\item[Origin] A single capability created at boot time in the bsp core.
  
\item[Retyability] No
  
\item[Mint parameters] Used to specify the region of io space the capability can access.
  \begin{itemize}
  \item Parameter 1: Start of the region
  \item Parameter 2: End of the region
  \end{itemize}
  
\item[Interpretation of rights] \note{Explain rights}
  
\item[Transferability to another core] Yes

\item[Last copy deleted] \note{NYI}
  
\item[Concrete representations] The in-memory representation on x86-64 is as follows:
  
  \begin{lstlisting}[language=Mackerel]
    datatype io_cap "IO capability" {
      start      16  "Start of the granted IO range";
      end        16  "End of the granted IO range";
    };
  \end{lstlisting}
\end{description}

\begin{invocation}{Outb}
  \arg IO port number
  \arg Output data
\end{invocation}
This invocation writes a byte to the the specified IO port

\begin{invocation}{Outw}
  \arg IO port number
  \arg Output data
\end{invocation}
This invocation writes a two byte word to the the specified IO port

\begin{invocation}{Outd}
  \arg IO port number
  \arg Output data
\end{invocation}
This invocation writes a four byte to the the specified IO port

\begin{invocation}{Inb}
  \arg IO port number
\end{invocation}
This invocation returns a byte read from the specified IO port.

\begin{invocation}{Inw}
  \arg IO port number
\end{invocation}
This invocation returns a 16-bit word read from the specified IO port.

\begin{invocation}{Ind}
  \arg IO port number
\end{invocation}
This invocation returns a 32-bit doubleword read from the specified IO port.

\subsection{IRQ table capability}
The IRQ table capability allows the holder to configure the user-level
handler dispatcher that will be invoked when the kernel receives
device interrupts.

\begin{description}
\item[Origin] Given to the first domain spawned on a core.
  
\item[Retyability] No
  
\item[Mint parameters] None
  
\item[Interpretation of rights] \note{Explain rights}
  
\item[Transferability to another core] No

\item[Last copy deleted] \note{NYI}
  
\item[Concrete representations] This capability type has no
  representation associated with it as it is used to simply give
  permissions to the holders and does not refer to any kernel data
  structure.
  \end{description}

\begin{invocation}{Set}
  \arg IRQ number
  \arg CSpace address of asynchronous endpoint capability
\end{invocation}
This invocation sets the user-level handler endpoint that will receive
a message when the given interrupt occurs.  While a handler is set,
interrupts will be delivered as IDC messages, as described in
\ref{sec:interrupts}.

\begin{invocation}{Delete}
  \arg IRQ number
\end{invocation}
This invocation clears the handler for the given IRQ.

\subsection{Kernel Capability}
So far, this capability is treated as the magic capability that gives
the holder a backdoor into performing special operations in the
kernel.

\begin{description}
\item[Origin] Given to the first domain spawned on a core.
  
\item[Retyability] No
  
\item[Mint parameters] None
  
\item[Interpretation of rights] \note{Explain rights}
  
\item[Transferability to another core] No

\item[Last copy deleted] \note{NYI}
  
\item[Concrete representations] The in-memory representation on x86-64 is as follows:
  
  \begin{lstlisting}[language=Mackerel]
    datatype kernel_cap "Kernel capability" {
      kernel_id   8  "Id of the Kernel";
    };
  \end{lstlisting}
\end{description}

\begin{invocation}{Spawn core}
  \arg Apic ID of the core to try booting
  \arg CSpace address of the RAM capability to use to relocate the new kernel
  \arg CSpace address of the Dispatcher capability of the first domain to run
  \arg Number of valid bits for the root CNode to associate with the Dispatcher capability
  \arg CSpace address of the root CNode to associate with the Dispatcher capability
  \arg CSpace address of the VNode to associate with the Dispatcher capability
  \arg CSpace address of the dispatcher frame to associate with the Dispatcher capability
\end{invocation}
The invocation requests the kernel to try booting another core.  The
kernel is to be relocated into the given memory region and to run the
the given domain.

\begin{invocation}{Get core ID}
  \arg None
\end{invocation}
The invocation returns the APIC ID of the core.

\begin{invocation}{Identify capability}
  \arg CSpace address of the capability to identify
  \arg Number of valid bits in the capability
  \arg Location of buffer to hold capability representation
\end{invocation}
The invocation stores the kernel's in-memory representation of the
capability into the given buffer.

\begin{invocation}{Identify CNode, get capability}
  \arg In memory representation of a CNode capability
  \arg Slot number of a capability within the CNode capability
  \arg Location of buffer to hold capability representation
\end{invocation}
The invocation stores the kernel's in-memory representation of the
capability located at the given slot in the given CNode into the given
buffer.

\begin{invocation}{Create capability}
  \arg In memory representation of a capability
  \arg CSpace address of the CNode the place the created capability in
  \arg Number of valid bits in the CSpace address of the CNode
  \arg Slot number to place the capability in
\end{invocation}
Creates the given capability in the given slot in the given CNode.

\begin{invocation}{Create capability}
  \arg In memory representation of a capability
  \arg CSpace address of the CNode the place the created capability in
  \arg Number of valid bits in the CSpace address of the CNode
  \arg Slot number to place the capability in
\end{invocation}
Creates the given capability in the given slot in the given CNode.

\note{The other invocations are outdated and will probably change
  when the monitors are discussed}
