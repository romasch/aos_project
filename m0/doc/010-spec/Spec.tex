%%%%%%%%%%%%%%%%%%%%%%%%%%%%%%%%%%%%%%%%%%%%%%%%%%%%%%%%%%%%%%%%%%%%%%%%%%
% Copyright (c) 2011, ETH Zurich.
% All rights reserved.
%
% This file is distributed under the terms in the attached LICENSE file.
% If you do not find this file, copies can be found by writing to:
% ETH Zurich D-INFK, Universitaetstr. 6, CH-8092 Zurich. Attn: Systems Group.
%%%%%%%%%%%%%%%%%%%%%%%%%%%%%%%%%%%%%%%%%%%%%%%%%%%%%%%%%%%%%%%%%%%%%%%%%%

\documentclass{scrreprt}
\usepackage[utf8]{inputenc}
\usepackage[T1]{fontenc}
\usepackage[pdftex]{graphicx}
\usepackage{times,helvet,url,color,listings,ctable}
\usepackage[pdftex]{hyperref}
\usepackage{cite} % must be after hyperref

% koma typearea: recalculate with new fonts
\typearea[current]{10}

% Comment shade color
\definecolor{boxshade}{gray}{0.85}

% hyperref setup
\definecolor{linkcol}{rgb}{0,0,0.7}
\hypersetup{
  pdftitle={Barrelfish Specification},
  plainpages=false,
  linktocpage,
  colorlinks,
  linkcolor=linkcol,citecolor=linkcol,pagecolor=linkcol,urlcolor=linkcol
  %breaklinks=true,pagebackref=true
}

% Default language for code listings is C
\lstset{
  language=C,
  basicstyle=\small,
  frame=lines,
  breaklines=true,
  showstringspaces=false,
  texcl=true,
  columns=flexible
}

\lstdefinelanguage{Mackerel}{
  morekeywords={datatype,device,register,regtype,constants,type,at,
              many,edit,io,also},
  sensitive=false,
  morecomment=[l]{//},
  morecomment=[s]{/*}{*/},
  morestring=[b]",
}

% sans-serif URLs
\urlstyle{sf}

% autoref (from hyperref) setup
\def\chapterautorefname{Chapter}
\def\sectionautorefname{Section}
\def\subsectionautorefname{Section}
\expandafter\def\csname section*autorefname\endcsname{Section}

\newcommand{\note}[1]{ [\textcolor{red}{\emph{#1}}]}
\newcommand{\todo}[1]{\note{\textbf{TODO:} #1}}

\newcommand{\noarginvocation}[1]{\paragraph{#1 invocation}}
\newcounter{invocArgCnt}
\newenvironment{invocation}[1]{%
  \noarginvocation{#1}
  
  \begin{list}{\emph{Argument~\arabic{invocArgCnt}:}}{%
    \usecounter{invocArgCnt}%
    \setlength{\rightmargin}{\leftmargin}%
    \setlength{\itemsep}{0ex}%
  }
  \renewcommand{\arg}{\item}
}{%
  \end{list}
}

% environment to keep track of the stuff I've clagged from the seL4 spec
\makeatletter
\newenvironment{sel4}{%
  \begin{lrbox}{\@tempboxa}\begin{minipage}{\columnwidth}
}{%
  \end{minipage}\end{lrbox}%
  \noindent\colorbox{boxshade}{\usebox{\@tempboxa}}
}\makeatother

\title{Barrelfish Specification}
\author{Andrew Baumann \and Simon Peter \and Timothy Roscoe \and
  Adrian Sch\"upbach \and Akhilesh Singhania}
\date{\input{specdate}}

\begin{document}
  \maketitle

  \chapter*{Acknowledgements}

  Paul, Rebecca, Tim, et al.

  \cleardoublepage
  \tableofcontents
  \listoftables
  \listoffigures
  %\lstlistoflistings

  \chapter{Introduction}

  Barrelfish is...

  \chapter{Barrelfish Kernel API}

  \section{System Calls}\label{sec:syscalls}

  \subsection{Invoke}\label{sec:sys_invoke}
  
  This system call takes at least one argument, which must the address of a
  capability in the caller's CSpace. The remaining arguments, if any, are
  interpreted based on the type of this first capability.
    
    Other than yielding, all kernel operations including IDC are
    provided by capability invocation, and make use of this call. The
    possible invocations for every capability type are described in
    the capability management document.
    
    This system call may only be used while the caller is
    \emph{enabled}. The reason is that the caller must be prepared to
    receive a reply immediately and that is only possible when
    enabled, as it requires the kernel to enter the dispatcher at the
    IDC entry point.
    
    \subsection{Yield}\label{sec:sys_yield}
    
    This system call yields the CPU. It takes a single argument, which must be
    either the CSpace address of a dispatcher capability, or \verb CPTR_NULL .
    In the first case, the given dispatcher is run unconditionally; in the
    latter case, the scheduler picks which dispatcher to run.
    
    This system call may only be used while the caller is \emph{disabled}.
    Furthermore, it clears the caller's \emph{disabled} flag, so the next time
    it will be entered is at the run entry point.
    
    \subsection{Debug system calls}

    The following debug system calls may also be supported, depending on
    build options, but are not part of the regular kernel interface.
    
      \subsubsection{No-op}
      This call takes no arguments, and returns directly to the
      caller. It always succeeds.
    
      \subsubsection{Print}
      This call takes two arguments: an address in the caller's vspace, which
      must be mapped, and a size, and prints the string found at that address
      to the console. It may fail if any part of the string is not accessible
      to the calling domain.
      
      \subsubsection{Reboot}
      This call unconditionally hard reboots the system.
      \note{This call should be removed -AB}

      \subsubsection{Debug}
      \todo{document me}

  \section{Dispatch and Execution}\label{sec:dispatch}

  A dispatcher consists of code executing at user-level and a data
  structure located in pinned memory, split into two regions. One
  region is only accessible from the kernel, the other region is
  shared read/write between user and kernel. The fields in the
  kernel-defined part of the structure are described in
  \autoref{tab:dispcb}.

  \ctable[
    caption=Dispatcher control structure,
    label=tab:dispcb,
    width=\textwidth
  ]{lll>{\raggedright}X}{}{
      \FL
      Field name & Size & Kernel R/W & Short description
      \ML
      \lstinline+disabled+ & word & R/W & If non-zero, the kernel will not
      upcall the dispatcher, except to deliver a trap.
      \NN
      \lstinline+haswork+ & pointer & R & If non-zero, the kernel will
      consider this dispatcher eligible to run.
      \NN
      \lstinline+crit_pc_low+ & pointer & R & Address of first instruction
      in dispatcher's critical code section.
      \NN
      \lstinline+crit_pc_high+ & pointer & R & Address immediately after
      last instruction of dispatcher's critical code section.
      \NN
      entry points & 4 function descriptors & R & Functions at which
      the dispatcher code may be invoked
      \NN
      \lstinline+enabled_save_area+ & arch specific & W & Area for kernel
      to save register state when enabled
      \NN
      \lstinline+disabled_save_area+ & arch specific & R/W & Area for
      kernel to save and restore register state when disabled
      \NN
      \lstinline+trap_save_area+ & arch specific & W & Area for kernel to
      save register state when a trap or a pagefault while disabled occurs
      \NN
      \lstinline+recv_cptr+ & capability pointer & R & Address of CNode to
      store received capabilities of next local IDC into
      \NN
      \lstinline+recv_bits+ & word & R & Number of valid bits within
      \lstinline+recv_cptr+
      \NN
      \lstinline+recv_slot+ & word & R & Slot within CNode to store
      received capability of next local IDC into
      \LL
  }

  Beyond these fields, the user may define and use their own data
  structures (eg. a stack for the dispatcher code to execute on,
  thread management structures, etc).

  \subsection{Disabled}

  A dispatcher is considered disabled by the kernel if either of the
  following conditions is true:

  \begin{itemize}
  \item its disabled word is non-zero
  \item its program counter is within the range specified by the
    \lstinline+crit_pc_low+ and \lstinline+crit_pc_high+ fields
  \end{itemize}

  The disabled state of a dispatcher controls where the kernel saves
  its registers, and is described in the following subsection. When
  the kernel resumes a dispatcher that was last running while
  disabled, it restores its machine state and resumes execution at the
  saved instruction, rather than upcalling it at an entry point.

  \subsection{Register save areas}

  The dispatcher structure contains enough space for three full copies
  of the machine register state to be saved. The \lstinline+trap_save_area+
  is used whenever the dispatcher takes a trap, regardless of whether
  it is enabled or disabled. Otherwise, the \lstinline+disabled_save_area+
  is used whenever the dispatcher is disabled (see above), and the
  \lstinline+enabled_save_area+ is used in all other cases.

  \autoref{fig:dispstatesaves} (Trap and PageFault states have
  been left out for brevity) shows important dispatcher states and into
  which register save area state is saved upon a state transition. The
  starting state for a domain is ``notrunning'' and depicted with a
  bold border in the Figure.

  \begin{figure}
    \centering
   \includegraphics[width=\textwidth]{disp_states_simple_save_area_analysis}
   \caption[Dispatcher state save areas]{Dispatcher state save areas.
     Trap and PageFault states
     omitted for brevity. Regular text and lines denote state changes
     by the kernel. Dashed lines and italic text denote state changes
     by user-space, which do not necessarily have to use the denoted
     save area. The starting state is in the bold
     node.}\label{fig:dispstatesaves}
  \end{figure}

  Arrows from right to left involve saving state into the labeled
  area. Arrows from left to right involve restoring state from the
  labeled area. It can be seen that no state can be overwritten. The
  kernel can recognize a disabled dispatcher by looking at the
  disabled flag, as well as the domain's instruction pointer. Nothing
  else needs to be examined.

  The dispatcher states are also depicted in \autoref{fig:dispstates}.

  \subsection{Dispatcher Entry Points}

  Unless restoring it from a disabled context, the kernel always
  enters a dispatcher at one of the following entry
  points. Whenever the kernel invokes a dispatcher at any of its entry
  points, it sets the disabled bit on. One (ABI-specific) register
  always points to the dispatcher structure. The value of all other
  registers depends on the entry point at which the dispatcher is
  invoked, and is described below.

  The entry points are:

  \begin{description}
  \item[Run] A dispatcher is entered at this entry point when it was
    not previously running, the last time it ran it was either enabled or
    yielded the CPU, and the kernel has given it the CPU. Other than the
    register that holds a pointer to the dispatcher itself, all other registers
    are undefined. The dispatcher's last machine state is saved in the
    \lstinline+enabled_save_area+.

  \item[PageFault] A dispatcher is entered at this entry point when it
    suffers a page fault while enabled. On entry, the dispatcher register is
    set, and the argument registers contain information about the cause of
    the fault. Volatile registers are saved in the
    \lstinline+enabled_save_area+; all other registers contain the user
    state at the time of the fault.

  \item[PageFault\_Disabled] A dispatcher is entered at this entry point when it
    suffers a page fault while disabled. On entry, the dispatcher register is
    set, and the argument registers contain information about the cause of
    the fault. Volatile registers are saved in the
    \lstinline+trap_save_area+; all other registers contain the user
    state at the time of the fault.

  \item[Trap] A dispatcher is entered at this entry point when it is
    running and it raises an exception (for example, illegal
    instruction, divide by zero, breakpoint, etc.). Unlike the other
    entry points, a dispatcher may be entered at its trap entry even
    when it was running disabled. The machine state at the time of the
    trap is saved in the \lstinline+trap_save_area+, and the argument
    registers convey information about the cause of the trap.

  \item[LRPC] A dispatcher is entered at this entry point when an
    LRPC message (see below) is delivered to it. This can only
    happen when it was not previously running, and was enabled. On
    entry, four registers are delivered containing the message payload,
    one stores the endpoint offset, and another contains the dispatcher pointer.
  \end{description}

  This diagram shows the states a \emph{dispatcher} can be in and how it
  gets there. The exceptional states Trap and PageFault have been
  omitted for brevity.

  \begin{figure}
    \centering
    \includegraphics[width=.7\columnwidth]{disp_states_simple}
    \caption[Typical dispatcher states]{Typical dispatcher states.
      Trap and PageFault states
      omitted for brevity. Regular text and lines denote state changes
      by the kernel. Dashed lines and italic text denote state changes
      by user-space. The starting state is in bold.}
    \label{fig:dispstates}
  \end{figure}

  \section{Inter-Dispatcher Communication}\label{sec:IDC}

  Inter-dispatcher communication (IDC) is a kernel-supported mechanism
  to allow dispatchers to communicate by sending messages. IDC is
  executed by invoking an IDC endpoint capability referring to a
  receiving dispatcher.

  \subsection{Endpoints}

  IDC communication takes place between dispatchers via endpoints.
  An endpoint is created by retyping a dispatcher capability
  into an IDC endpoint capability. It refers to exactly one dispatcher, and
  to one endpoint buffer structure within that dispatcher.
  An endpoint buffer is a kernel-specified data structure located within the
  dispatcher frame, where the kernel delivers IDC messages.

  The kernel guarantees messages to either be delivered to the
  receiving dispatcher or to return to the sender with an error status
  code in the event that the receiver is unable to receive the message.
  This implies that messages are never dropped silently by the kernel but does
  not guarantee that messages are never dropped on the whole communication path,
  which involves the receiving dispatcher.

  It should be noted that endpoint capabilities may be freely copied, and do not
  uniquely identify a sender. An endpoint capability can be
  transferred to several dispatchers, all of whom may use the same
  endpoint and thus the same buffer when sending messages.

  \subsection{Message Transfer}

  To send IDC, a dispatcher invokes an endpoint capability to the
  receiving dispatcher. The message it wishes to send is provided as
  argument to the invocation, as well as flags to specify additional
  parameters influencing the message transfer.

  An IDC message is delivered by the kernel allocating space in the receiver's
  endpoint buffer, and writing the message contents. The receiver must poll its
  endpoints to detect incoming messages, and consume them in order to free space
  in the endpoint buffer for new messages. \todo{detail!}

  A dispatcher that executes an IDC invocation is considered to have
  yielded the CPU while enabled. Therefore, the next time it is
  entered may be either at the \textbf{Run} or \textbf{LRPC} entry
  points.

  \subsection{Capability transfer}

  IDC can also be used to transfer capabilities from the sending dispatcher's
  domain to a receiving dispatcher's domain.\todo{document cap transfer!}

  \subsubsection{Flags}

  If the ``sync'' flag is set and the message transfer succeeds, the kernel
  will immediately dispatch the receiver. Effectively, the sender yields to the
  receiver.

  If the ``yield'' flag is set, and the message fails for one of the following reasons:
  \begin{itemize}
   \item the receiver's message buffer is full
   \item the sender specified a capability, but it cannot be delivered because the 
     receiver's capability receive slot is non-empty
  \end{itemize}
  \ldots then the kernel will also immediately dispatch the receiver (without 
  performing a message transfer). Again, in this case the sender effectively yields
  to the receiver.

  \subsubsection{LRPC}

  \todo{document!}

  In this mode of IDC, the kernel performs a controlled
  context switch from the sending to the receiving dispatcher,
  preserving the capability invocation register state which is used to
  deliver the message. The sender dispatcher is not blocked, however
  it implicitly donates the remainder of its timeslice to the
  receiver.

  If the receiving dispatcher is disabled, and the ``yield'' flag was set,
  the kernel sets the return
  register in the sending dispatcher's \lstinline+enabled_save_area+
  to \lstinline+SYS_ERR_TARGET_DISABLED+. The kernel then switches to
  and resumes the target dispatcher. In effect, an LRPC
  operation when the target is disabled becomes a directed yield of
  the CPU to the target dispatcher. If the ``yield'' flag was not set,
  the kernel simply returns the same error code to the sender and runs the
  sender.

  \subsection{Interrupt delivery}\label{sec:interrupts}

  Hardware interrupts are delivered by the kernel as asynchronous IDC
  messages to a registered dispatcher. A dispatcher can be registered
  as for a specific IRQ by invoking the IRQTable capability,
  passing it an IDC endpoint to the dispatcher and the IRQ
  number. It is not possible for multiple IDC endpoints to be
  registered with the same IRQ number at any one time.

  Henceforth, the kernel will send an IDC message using asynchronous
  delivery to the registered endpoint. Asynchronous
  IDC is used as it does not cause priority inversion by directly
  dispatching the target dispatcher.

  Refer to \autoref{apx:arch_specific} for more information about
  valid hardware interrupts for an architecture-specific
  implementation of Barrelfish.

  \subsection{Exception delivery}

  When a CPU exception happens in user-space, it is reflected to the
  dispatcher on which it appeared. Page
  faults are dispatched to the pagefault entry point of the
  dispatcher. All other exceptions are dispatched to the trap entry
  point of the dispatcher. The disabled flag of the dispatcher is
  ignored in all cases and state is saved to the trap save area.

  \section{Virtual Memory}

  \todo{Our memory model is based on capabilities and is quite similar to seL4.}

  \section{Initial Address Space}

  Our address space initialization is similar to the one of seL4, but
  we do not follow their boot protocol to the word. Here is our
  version:

  We have a special program called \lstinline+init+ that is run by the kernel
  after bootup as an ELF64 executable. In order to function, it has to
  receive some information by the kernel. We show first how it
  receives this information from its (\lstinline+init+'s) own perspective and
  then how the kernel gathers and transmits this information to
  \lstinline+init+.

  \subsection{User-Space Perspective}

  \lstinline+init+'s virtual address space size at startup is at most 4 MBytes
  (the amount of pagetable kernel memory left for it), mapped as 4K
  pages, starting from \lstinline+0x200000+ (2 Meg). \lstinline+init+'s text/data segments
  should be aligned consecutively and start at \lstinline+0x400000+ (4 Meg),
  leaving it 2 MBytes for its text and data.

  We have a \lstinline+bootinfo+ structure, shown in \autoref{lst:bootinfo}.

\begin{lstlisting}[float,caption={\lstinline+bootinfo+ structure},
  label=lst:bootinfo]
struct bootinfo {
    // Base address of small memory caps region
    capaddr_t             small_untyped_base;
    // Number of small memory caps
    size_t              small_untyped_count;
    // Base address of large memory caps region
    capaddr_t             large_untyped_base;
    // Number of large memory caps
    size_t              large_untyped_count;
    // Number of entries in regions array
    size_t              regions_length;
    // Memory regions array
    struct mem_region   regions[MAX_MEM_REGIONS];
};
\end{lstlisting}

  This structure is mapped into \lstinline+init+'s virtual memory at
  address \lstinline+0x200000+ (2 Meg) and is at most a 4K page in
  size. \lstinline+small_untyped_base+ points to the capability to the
  CNode, holding a number (given by \lstinline+small_untyped_count+) of
  small untyped capabilities. These can be used for easy setup of
  init's own address space. \lstinline+large_untyped_base+ and
  \lstinline+large_untyped_count+ is similar for (much) larger untyped
  capabilities. Their sizes can be found in the \lstinline+regions+ array,
  of size \lstinline+regions_length+ entries. An entry is defined by a
  \lstinline+mem_region+ struct, shown in \autoref{lst:mem_region}.

\begin{lstlisting}[float,caption={\lstinline+mem_region+ structure},
  label=lst:mem_region]
struct mem_region {
    paddr_t             base;   // Address of the start of the region
    size_t              size;   // Size of region in bytes
    enum region_type    type;   // Type of region
    uint64_t            data;   // Additional data, based on region type
};
\end{lstlisting}

  Its fields should be self-explanatory. The possible region types are
  defined by \lstinline+enum region_type+, shown in \autoref{lst:region_type}.

\begin{lstlisting}[float,caption={\lstinline+region_type+ enumeration},
  label=lst:region_type]
enum region_type {
    RegionType_Empty,           // Empty memory
    RegionType_InitCaps,        // init's caps mapped here
    RegionType_RootTask,        // Code/Data of init itself
    RegionType_Device,          // Memory-mapped device
    RegionType_CapsOnly         // Kernel-reserved memory
};
\end{lstlisting}

  These are the same as those in seL4.

  \subsubsection{Initial Capability Address Space}

  \lstinline+init+'s initial CSpace is shown in \autoref{fig:init_cspace}.

  \begin{figure}
    \centering
    \includegraphics[width=\textwidth]{init_cspace}
    \caption{\texttt{init}'s initial capability space layout}
    \label{fig:init_cspace}
  \end{figure}

  \subsection{Kernel Perspective}

  In the following, 'cn' will be short for CNode, \lstinline+init_dcb+ is short
  for \lstinline+init+'s DCB, \lstinline+replyep+ is short for \lstinline+init+'s system call reply
  endpoint. The kernel sets up \lstinline+init+'s domain as follows:

  \begin{itemize}
  \item It allocates physical pages for: rootcn, taskcn, smallcn,
    supercn, \lstinline+init_dcb+, replyep.
  \item Map bootinfo, \lstinline+init_dcb+, replyep, rootcn, pml4, pdpt, pdir and
    ptables in that order.
  \item Allocate 64 physical pages and put untyped caps to them into
    smallcn.
  \item Map taskcn, smallcn, supercn, in that order.
  \item Setup \lstinline+init+'s DCB.
  \item Load \lstinline+init+ ELF64 binary into memory, map memory and allocate
    caps.
  \item Add all remaining memory as untyped caps to power of two large
    regions into supercn. They may not be more than 64.
  \item Fill \lstinline+bootinfo+ struct.
  \item \lstinline+schedule()+ init.
  \end{itemize}

  \section{Scheduling}

  Upon reception of a timer interrupt, the kernel calls `schedule()`,
  which selects the next dispatcher to run. At the moment, a simple
  round-robin scheduler is implemented that walks a circual
  singly-linked list forever.

  \section{TODO}

  \begin{itemize}
  \item virtual machine support
  \item timers
  \item resource management
  \item thread migration
  \item event tracing / performance monitoring
  \end{itemize}

  \chapter{Barrelfish Library API}

  \todo{documentation of libbarrelfish}

  \section{Capabilities}
  
  \subsection{Data types}
  
  cap\_info
  cnode\_info
  
  get\_cap\_valid\_bits
  get\_cap\_addr
  get\_cnode\_valid\_bits
  get\_cnode\_addr
  build\_cnode\_info ?
  
  \subsection{Functions}

  cap\_copy
  cap\_mint
  cap\_retype
  cnode\_create
  cnode\_create\_raw ?
  
  ram\_alloc
  
  async\_endpoint\_create
  local\_endpoint\_create

  \subsection{Invocations}
  invoke\_*
  
  \subsection{Syscalls}
  
  syscall
  sys\_yield
  cap\_invoke
  cap\_invoke\_wait
  
  
  \section{VSpace management}
  
  struct vnode
  struct vlist
  
  vspace\_alloc
  vspace\_map
  vspace\_map\_raw ?
  vspace\_free
  vspace\_alloc\_map
  vspace\_map\_attr
  vspace\_map\_attr\_raw ?

  \section{Dispatch and threading}

  \section{Spawning domains}
  \subsection{Initial capability space}

  The initial capability space of other domains is similar, but lacks the other
  cnodes in the root cnode, as illustrated in \autoref{fig:app_cspace}.

  \begin{figure}
    \centering
    \includegraphics[width=\textwidth]{app_cspace}
    \caption{initial capability space layout of user tasks}
    \label{fig:app_cspace}
  \end{figure}


  \appendix
  \chapter{Glossary}

  \begin{description}
  \item[Capability] Every kernel object is represented by a capability,
    allowing the user who holds that capability to manipulate it. We use
    partitioned capabilities: capabilities are stored in memory accessible only
    to the kernel, and are manipulated or invoked through the use of addresses
    in the CSpace.

  \item[CSpace] The capability address space, in which all capabilities reside,
    is constructed and managed by user-space code through a hierarchy of page
    table-like structures, called CNodes. The protection domain of user code
    is determined by the capabilities existing in its CSpace.

  \item[VSpace] The virtual address space

  \item[Dispatcher] Kernel-scheduled entity, responsible for
    scheduling/managing the execution of user code. Dispatchers are identified
    by DCB capabilities. Every dispatcher has a CSpace and VSpace pointer,
    which determine its protection domain and virtual address space. Multiple
    dispatchers may share a CSpace or VSpace.

  \item[Domain] Although not directly part of the Barrelfish kernel API, the
    word domain is used to refer to the user-level code sharing a protection
    domain and (usually) an address space. A domain consists of one or more
    dispatchers.

  \item[DCB] Dispatcher control block, the kernel object associated with
    a dispatcher, and therefore also one of the system's capability types.
    \note{I'd prefer to avoid this term, as it can be confusing. -AB}

  \item[IDC] Inter-dispatcher communication, the kernel-mediated message-passing
    primitive. There are two types of IDC: the general case of
    \emph{asynchronous IDC}, and an optimised \emph{local IDC} variant possible
    only when both the sender and receiver execute on the same core.

  \item[Endpoint] A type of capability that, when invoked, performs an IDC.
    There are two endpoint types (asynchronous and local) to match the two
    types of IDC.

  \item[Channel] A uni-directional kernel-mediated communication path
    between dispatchers. All messages travel over channels. Holding a
    capability for a channel guarantees the right to send a message to it
    (although the message may not be sent for reasons other than
    protection).

    \item[Mapping Database] The mapping database is used to facilitate
      retype and revoke operations.

      A capability that is not of type dispatcher, can only be retyped once.
      The mapping database facilitates this check.

      When a capability is revoked, all its descendants and copies are deleted.
      The mapping database keeps track of descendants and copies of a capability
      allowing for proper execution of a revoke operation.

      Each core has a single private mapping database.
      All capabilities on the core must be included in the database.

  \item[Descendant] A capability X is a descendant of a capability A if:

    \begin{itemize}
      \item X was retyped from A,
      \item or X is a descendant of A1 and A1 is a copy of A,
      \item or X is a descendant of B and B is a descendant of A,
      \item or X is a copy of X1 and X1 is a descendant of A.
    \end{itemize}

  \item[Ancestor] A capability A is an ancestory of a capability X
    if X is a descendant of A.
  \end{description}

  \chapter{Implementation}

  This chapter covers the implementation and algorithm of some subsystems.

  \section{Mapping Database}

  This section describes the mapping database is more detail.
  It covers the algorithms including implementation details
  and invariants on the database.

  \subsection{Implementation details}

  The database implements the following functions:
  \begin{itemize}
    \item \emph{is copy} Checks if two capabilities are copies of each other.
      Two capabilities are copies if they are of the same type
      and they refer to same kernel object.
      PhysAddr, RAM, Frame, DevFrame, CNode, Dispatcher, Kernel, EndPoint
      Capability types explicitly reference kernel objects
      so capabilities of such types can be tested simply.
      We cannot handle other capability types yet,
      comparing two VNode, we always return false
      and comparing two IO or IRQTable, we always return true.

    \item \emph{is ancestor} Checks if one capability is a parent of another.
      In our current implementation, some capability types cannot have descendants
      and some capability types cannot have ancestors.
      For the rest, we check if the parent child relationship is possible
      based on the retyping type allowed
      and check if the kernel object the child refers is inclusive
      in the range of kernel objects the parent refers to.

    \item \emph{has descendants} Checks if a capability has any descendants
      Walks the entire database checking if the capability has any descendants.
      The function returns true when the first descendant is found
      and if a capability other than a copy is found, it returns false.

    \item \emph{has copies} Checks if a capability has any copies
      Walks the entire database checking if the capability has any copies.
      The function returns true when the first copy is found
      and if a capability other than a descendant is found, it returns false.

    \item \emph{insert after} Inserts a set of contiguous capabilities
      after the given capability

    \item \emph{insert before} Inserts a set of contiguous capabilities
      before the given capability

    \item \emph{set init mapping} Inserts a capability into the database
      in the appropriate location.
      If any copies or ancestors of the capability exist,
      the capability is inserted after a copy or after the closest ancestor.
      If no relatives exist in the database,
      the capability is inserted at the top of the database.

    \item \emph{remove mapping} Removes the capability from the database
  \end{itemize}

  \subsection{Invariants}

  Some invariants on the database that must be true at all times.
  \begin{enumerate}
  \item The next and prev pointers on a capability are never NULL.
  \item There is only one database per core.
    Any capability on a core can be reach from another on the core.
  \item Two separate databases do not share any capabilities.
    The set of next and prev pointers on one database
    is disjoint from the set of next and prev pointers on another.
  \item Each capability on a core is on the database of the core.
    A capability will eventually be visited by starting at any other capability
    and walking the database.
  \item The database is circular.
    Walking in either direction from any capability,
    the same capability will eventually be reached again.
  \item The head of the database cannot have any ancestors.
  \end{enumerate}

  \subsection{Current Limitations}

  The database has the following limitations.
  \begin{enumerate}
  \item It does not handle VNode capabilities and other types.
    Other types are not as crucial, but VNode will become a priority shortly.

  \item No indexing for quickly inserting brand new capabilities.
    The implementation starts at the head of the database
    and traverses the entire database till an appropriate location is found.

  \item Not necessarily a limitation but an important note.
    The current implementation can report certain capabilities as descendants
    when one could have been created by copying another
    and report a descendant as a copy when it was created by retyping the ancestor.
    This requires the implementation of some functions to test for relationships
    in the correct order.
    This may lead to some unforeseen issues later
  \end{enumerate}

  \chapter{Architecture-Specific Features}\label{apx:arch_specific}

  This chapter covers features specific to one implementation of
  Barrelfish on a specific hardware architecture.

  \section{x86-64}

    The x86-64 implementation of Barrelfish is specific to the AMD64
    and Intel 64 architectures. This text will refer to features of
    those architectures. Those and further features can be found in
    \cite{intelsa} and \cite{amdsa} for the Intel 64 and AMD64
    architectures, respectively.

    \subsection{VSpace}

    The page table is constructed by copying VNode capabilities into VNodes to
    link intermediate page tables, and minting Frame / DeviceFrame capabilities 
    into leaf VNodes to perform mappings.

    When minting a frame capability to a
    VNode, the frame must be at least as large as the smallest page size. The
    type-specific parameters are:

    \begin{enumerate}
      \item \textbf{Access flags:}
        The permissible set of flags is PTABLE\_GLOBAL\_PAGE
        | PTABLE\_ATTR\_INDEX | PTABLE\_CACHE\_DISABLED |
        PTABLE\_WRITE\_THROUGH. Access flags are set from frame capability
        access flags. All other flags are not settable from user-space (like
        PRESENT and SUPERVISOR).

      \item \textbf{Number of base-page-sized pages to map:} If non-zero, this
        parameter allows the caller to prevent the entire frame capability from
        being mapped, by specifying the number of base-page-sized pages
        of the region (starting from offset zero) to map.
    \end{enumerate}

    \subsection{IO capabilities}

    IO capabilities provide kernel-mediated access to the legacy IO space of
the processor. Each IO capability allows access only to a specific range of
ports.

    The Mint invocation (see \autoref{sec:mint}) allows the permissible
port range to be reduced (with the lower limit in the first
type-specific parameter, and the upper limit in the second type-specific
parameter).

    At boot, an IO capability for the entire port space is passed to the
initial user domain. Aside from being copied or minted, IO capabilities may not
be created.

  \subsection{Interrupts and Exceptions}

  \subsubsection{Interrupts}

  The lower 32 interrupts are reserved as CPU exceptions. Thus, there
  are 224 hardware interrupts, ranging from IRQ number 32 to 255.
  
  The kernel delivers an interrupt that is not an exception and not
  the local APIC timer interrupt to user-space. The local APIC timer
  interrupt is used by the kernel for preemptive scheduling and not
  delivered to user-space.

  \subsubsection{Exceptions}

  The lower 32 interrupts are reserved as CPU exceptions. Except for a
  double fault exception, which is always handled by the kernel
  directly, an exception is forwarded to the dispatcher handling the
  domain on the CPU on which it appeared.

  Page faults (interrupt 14) are dispatched to the `pagefault` entry
  point of the dispatcher. All other exceptions are dispatched to the
  `trap` entry point of the dispatcher.

\bibliographystyle{plain}
\bibliography{defs,barrelfish}

\end{document}
